\documentclass[letterpaper,12pt]{cicese}
	\usepackage[letterpaper,left=2.5cm,right=2.5cm,top=3cm,bottom=3cm]{geometry}
	\usepackage{setspace}
	\usepackage{natbib}
	\usepackage{graphicx}
	\usepackage{float}
	\usepackage{caption}
	\begin{document}
	\doublespace
	\title{Detecci\'on de Ansiedad Por Medio de C\'omputo Vestible en Cuidadores de Pacientes con Problemas Mentales}
	\author{Dari\'en Alberto Miranda Boj\'orquez}
	\maketitle
	\newpage
	\tableofcontents
	\newpage

		\chapter{Introducci\'on} 
			%TODO: HABLAR DE ANSIEDAD
			%El estr\'es es un fen\'omeno que la poblaci\'on de nuestra sociedad moderna experimenta cotidianamente. S\'olo en Estados Unidos, tres cuartos de sus
			%habitantes experimentan s\'intomas relacionados con el estr\'es \citep{Lu2012}. El estr\'es se demuestra de diferentes maneras en las personas tanto psicol\'ogica
			%como fisiol\'ogicamente. Los efectos psicol\'ogicos incluyen: ansiedad, depresi\'on, desgaste, insomnio e insatisfacci\'on \citep{Sebastian2013556}. La relaci\'on que tiene
			%con la ansiedad es que la ansiedad es la se\~nal psicofisiol\'ogica de que la respuesta al estr\'es ha sido iniciada.\citep{PMID2235645}
			
			
			%Un cierto  nivel de estr\'es es necesario para lograr las tareas de los trabajos de nuestra sociedad moderna. Nos impulsa a completar tareas basadas en 
			%calendarios. Incluso, \emph{``El estr\'es puede no ser observado como un problema por las personas, niveles altos de estr\'es  son percibidos com\'unmente 
			%como una norma, una se\~nal de que est\'an  haciendo su mejor esfuerzo para completar objetivos"}\citep{Bakker2012SMS}. Durante una situaci\'on de estr\'es, el cuerpo se encuentra
			%tensando los m\'usculos mientras que el sistema nervioso parasim\'etrico trabaja para controlar este problema balanceandolos para lograr una homeostasis\citep{Ayzenberg2012}. Sin
			%embargo, periodos largos en este estado pueden llevar a problemas de salud como dolores de cabeza, fatiga, ansiedad y depresi\'on.
			%como puede llegar a ser un desorden? (post traumatic disorder)
			%computo vestible
			%TODO:Que es la ansiedad?
			La ansiedad es una emoci\'on caracterizada por sensaciones de tensi\'on, pensamientos de preocupaci\'on y cambios f\'isicos como incremento en la presi\'on arterial.\citep{psychologyapa}.Es com\'un que la poblaci\'on en general tenga episodios de ansiedad debido a los tipos de trabajo de nuestra sociedad moderna. Durante esos lapsos de tiempo, la persona suele experimentar un nivel de ansiedad el cual es una reacci\'on normal para lograr objetivos. Sin embargo, cuando la persona experimenta un nivel de ansiedad el cual es tan alto que no le permite manejar su vida normal, se dice que la persona tiene un desorden de ansiedad\citep{repetto2013}.

			Uno de los sectores de poblaci\'on vulnerables, son los cuidadores de pacientes con autismo o demencia. Se encuentra documentado que los cuidadores, al llevar una carga f\'isica, cognitiva y emocional derivada de su labor les genera padecimientos como ansiedad, estr\'es, y hasta la muerte\citep{Chen2013}. Debido a que los cuidadores no necesariamente son personas con una formaci\'on profesional, estos efectos pueden verse aumentados. Por lo general, los cuidadores que son familiares del paciente son a\'un mas afectados debido a que necesitan administrar el tiempo de trabajo, familia, actividades sociales y la actividad misma del cuidado del paciente.

			%autismo
			El desorden de autismo (referido como autismo) es una de las variedades del Espectro de Des\'ordenes del Autismo (ASD) y est\'a caracterizado por interacciones sociales da\~nadas, ausencia de habilidades de comunicaci\'on, movimientos estereotipados y mal comportamiento en general\citep{bernier2010autism}. Existen escuelas especiales para ni\~nos con este padecimiento. 

			%demencia
La demencia es un s\'indrome del declive de las habilidades cognitivas. Los s\'intomas comunes son: problemas de memoria, dificultades para realizar tareas familiares, mal juicio, deterioro del lenguaje hablado y cambios de humor\citep{Aziz}. Afecta alrededor de el 4\% de las personas mayores de 65 a\~nos y al 40\% de las personas mayores de 90. 

			Por otro lado, el c\'omputo vestible nos permite llevar computadoras con nosotros de la misma manera que llevamos la ropa puesta. Al ``vestir" un dispositivo,
			el usuario tiene acceso a una computadora que es capaz de monitorearlo a \'el y a su entorno por medio de sensores. Dichos sensores pueden medir entre
			otras cosas: movimientos del cuerpo del usuario, la posici\'on del usuario, intensidad de luz, ruido, im\'agenes de su ambiente, ritmo card\'iaco, capacidad
			conductiva de la piel, distancias, entre otros. Debido a su caracter\'istica de ser vestible, se pueden hacer monitoreos constantes y mas precisos que con
			los sistemas tradicionales y ayudar en las tareas de la vida cotidiana.
	
			El uso de c\'omputo vestible para la detecci\'on de la ansiedad abre una posibilidad para ayudar a reducir el riesgo a la salud mental
			de los cuidadores de pacientes con autismo o demencia. La siguiente secci\'on ejemplifica posibles escenarios de aplicaciones reales.
			\section{Escenarios de aplicaci\'on}
				\begin{enumerate}
					\item \emph{Cuidadores de pacientes con autismo.}
					Mar\'ia es una maestra novata de este tipo de escuelas. Entre sus labaores, se encuentra
					atender a un grupo de mas de diez alumnos. Tratar con los ni\~nos es una labor muy exigente y estresante. Si Mar\'ia llega al l\'imite de su capacidad
					de carga de tareas, su calidad como maestra y su estado de salud puede verse disminu\'ido temporalmente. Juliana, la encargada del personal de la
					escuela debe ser capaz de saber si Mar\'ia est\'a siendo afectada severamente y pasar la carga a otra maestra mas desocupada o bien, con mas experiencia. Mar\'ia
					viste unos lentes intelegintes que incluye sensores que infieren el estado de ansiedad en ella y que adem\'as registra por medio de video y su
					frecuencia card\'iaca los eventos de ansiedad. El dispositivo indica a Juliana el nivel de ansiedad y le permite hacer el cambio de maestra. Los episodios son registrados
					en un sistema en la web con la fecha, hora, severidad y videoclip del evento. Luego,
					en una reuni\'on, Juliana utiliza los videos para darle consejos a Mar\'ia sobre como controlar la situaci\'on, relajarse por medio de t\'ecnicas
					y ganar experiencia.
					\item \emph{Cuidadores de pacientes con demencia.}
						Valeria tiene un padre con demencia. Le toma una buena parte del d\'ia ayudarlo
						con tareas cotidianas como ayudarle a lavarse los dientes, cambiarse, proporcionarle sus medicinas entre
						otras tareas. Su padre, presenta comportamientos conflictivos, como preguntarle constantemente la misma pregunta, deambular dentro y fuera de la casa. Esto le genera una alta cantidad de ansiedad, pasando por lapsos donde su ritmo card\'iaco y de respiraci\'on se encuentran muy alterados y no le permite pensar
						claramente. Un dispositivo vestible que detecta los episodios de ansiedad, le da recomendaciones sobre como tranquilizarse a trav\'es de
						sencillos ejercicios de respiraci\'on.
				\end{enumerate}

		\chapter{Marco te\'orico} 
			A continuaci\'on, se definen los conceptos al rededeor de la naturaleza y detecci\'on de la ansiedad.
			%stimuli
			%\section{Estr\'es}
			%	El estr\'es puede ser visto como una carga mental o \emph{stimuli} el cual tiene un peso asociado. Este peso puede ser positivo o negativo dependiendo
			%	de la apreciaci\'on del sujeto. Cuando percibimos un \emph{stimulus} o un grupo de {stimuli} como amenzante, lo especificamos como \emph{estr\'es}\citep{Sebastian2013556}.
			%	Este es un tipo de experiencia que puede ser f\'acilmente cuantificada por medio de instrumentos psicol\'ogicos.
				
			%	El modelo descrito por Levine \citep{Sebastian2013556} explica el proceso del estr\'es de la siguiente manera: La carga, que incluye los factores estresantes y el
			%	stimuli es evaluada por el cerebro. Despu\'es de la evaluaci\'on, puede haber una respuesta al estr\'es, la cual funciona como alarma para el cerebro.
			%	El cerebro puede entonces modificar el stimuli o la percepci\'on del stimuli por medio de acciones y periodos de inactividad. Por \'ultimo, la respuesta
			%	fisiol\'ogica puede generar tensi\'on  o entrenamiento, dependiendo de la actividad. Un estr\'es sostenido puede llevar a una patolog\'ia (tensi\'on).
			\section{Caracterizaci\'on fisiol\'ogica}
				Los m\'etodos comunes para la detecci\'on de la ansiedad  por medio de se\~nales fisiol\'ogicas son: 
					\begin{itemize}
						\item \emph{Frecuencia Card\'iaca (HR):} Normalmente, suponen que el ritmo card\'iaco aumenta durante los periodos de ansiedad. 
						\item \emph{Respuesta Galv\'anica de la piel (GBR):} De la misma manera que el ritmo card\'iaco, la respuesta galv\'anica de la piel suele cambiar durante los periodos de ansiedad.
						\item \emph{An\'alisis del habla:} Algunas sutilezas del habla suelen notarse en los periodos de ansiedad, tales como tartamudeo o tonos de voz.
						El procesamiento del habla puede  ser utilizado como herramienta de medici\'on de ansiedad.
					\end{itemize}
			\section{C\'omputo vestible}
				El c\'omputo vestible es aquel en el que la computadora es lo suficientemente peque\~na para poder ser "vestida" como ropa mientras que asiste en las
				tareas cotidianas de la vida del usuario\citep{Starner97augmentedreality}. Algunos lugares t\'ipicos del cuerpo donde son usados son: los ojos, o\'idos, brazos, piernas o torso.
				Los individuos que usan estos dispositivos suelen cargarlos f\'acilmente con ellos por largos periodos de tiempo durante el d\'ia. Al tener sensores
				especializados, y si consideramos que el usuario puede llevar mas de uno puesto, tenemos una herramienta de sensado poderosa que es capaz de obtener
				informaci\'on del cuerpo del individuo al mismo tiempo que provee una comunicaci\'on m\'as natural que aquella del c\'omputo tradicional del escritorio.

			\section{Contexto}
				El contexto es definido por Dey \citep{Dey2001} como toda aquella informaci\'on que puede ser utilizada para caracterizar la situaci\'on de una entidad. Donde una
				entidad puede ser una persona, lugar u objeto computacional. El tener informaci\'on contextual permite a los programas funcionar de una manera adaptativa,
				en la que toma en cuenta la situaci\'on y act\'ua acorde a ella. Un ejemplo de c\'omputo contextual que usamos diaramente es el tel\'efono inteligente.
				Si el ambiente del usuario tiene mucha luz, el tel\'efono decide reducir el brillo de la pantalla, o bien, si se est\'a haciendo una llamada en un lugar muy ruidoso,
				decide aplicar filtros de ruido para mejorar la calidad de la voz.
		
			\section{C\'omputo vestible consciente del contexto}
				Al encontrarse cerca del usuario mientras ayuda en las tareas diarias, el c\'omputo vestible debe de hacer uso del contexto para ayudar de una manera
				que sea significante para el usuario. De no ser as\'i, puede generar frustraci\'on y desuso. Algunas de las caracter\'isticas del c\'omputo vestible
				con respecto al contexto que deben de tener estos dispositivos seg\'un \citep{Rhodes97thewearable} son las siguientes:
				\begin{itemize}
					\item{\emph{Port\'atil y al mismo tiempo operacional:}} Una computadora vestible es capaz de ser usada mientras que el usuario se encuentra
					en movimiento. Al estar en movimiento, su contexto es much mas din\'amico: Cambia a nuevos espacios f\'isicos, encuentra nuevos objetos 
					y gente (entidades).Los servicios e informaci\'on que requiere cambiar\'an en base a las nuevas entidades.
				\end{itemize}
				\begin{itemize}
					\item{\emph{Uso en modo manos libres:}} Una computadora vestible tiene la intenc\'ion de ser operada con m\'inimo uso de las manos,
					bas\'andose en la entrada por voz o controles con una sola mano. Limitar el uso de los mecanismos de entrada incrementa la necesidad
					de obtener informaci\'on contextual impl\'icitamente sensada.
				\end{itemize}
				\begin{itemize}
					\item{\emph{Sensores:}} Para disminuir la entrada expl\'icita del usuario, una computadora vestible deber\'ia de utilizar sensores para
					colectar informaci\'on acerca del ambiente del usuario. La informaci\'on obtenida directamente por los sensores en el cuerpo del
					usuario puede ser combinada con sensonres puestos en el ambiente en aplicaciones reales.
				\end{itemize}
				\begin{itemize}
					\item{\emph{Pro activo:}} Una computadora vestible debe de actuar en base al comportamiento del usuario incluso cuando el usuario no
					est\'a expl\'icitamente utiliz\'andolo. Esta es le esencia de la computaci\'on basada en el contexto: la computadora analiza el
					contexto de usuario y provee de tareas y servicios relevantes a las actividades del usuario interrumpiendolo s\'olo cuando es apropiado.
				\end{itemize}
				\begin{itemize}
					\item{\emph{Siempre encendido:}} Una computadora vestible siempre est\'a encendida. Esto es importante para el c\'omputo consciente del contexto
					porque la computadora vestible debe de monitorear constantemente la situaci\'on del usuario para que se pueda adaptar y responder
					adecuadamente. Debe de ser capaz de proveer servicios \'utiles al usuario en cualquier momento.
				\end{itemize}
				%como se usa todo esto para dar soporte a las actividades de la vida diaria?

		\chapter{Trabajo previo}
				Diferentes trabajos se han realizado para la detecci\'on de la ansiedad y fen\'omenos similares. A continuaci\'on se listan algunos de los mas relevantes:
				\begin{itemize}
					\item{\emph{AutoSense}\citep{Ertin2011}} es un dispositivo especialmente dise\~nado para el sensado del estado del estr\'es del usuario. Posee seis diferentes
					sensores que pueden colectar informaci\'on cardiovascular, respiratoria, t\'ermica, de respuesta galv\'anica de la piel y de acelerometr\'ia. Los
					datos son luego enviados por medio de una conexi\'on bluetooth a una aplicaci\'on en Android donde se realizan inferencias y provee una capa de 
					servicio a otras aplicaciones.
		
					\item{\emph{FaceIt}\citep{Rennert2013}} es una herramienta que permite detectar, almacenar y recordar situaciones en las que el usuario presente ansiedad. Utiliza
					de base la Terapia de Comportamiento Cognitiva (CBT por sus siglas en ingl\'es). Un dispositivo vestible tipo Memoto que cuelga del cuello del usuario
					detecta por medio del ritmo card\'iaco los episodios de ansiedad y registrs solo en esos momentos video, audio y la localizaci\'on geogr\'afica. Posteriormente,
					los datos son transmitidos a internet, y una p\'agina web les permite a los pacientes almacenar sus periodos y revivir las situaciones de manera
					controlada. La investigaci\'on indica que: \emph{``las aplicaciones m\'oviles tienen la capacidad de incrementar
					la autoconciencia y reducir los niveles de estr\'es"}. de manera que \emph{``Por estas razones, creemos que este enfoque tiene el potencial de asisistir
					el tratamiento de comportamiento cognitivo para aquellos con ansiedad social"}.

					\item{\emph{Stress@work}\citep{Bakker2012SMS}} es un marco de trabajo para la medici\'on, entendimiento y predicci\'on y manejo del estr\'es. Utiliza la Respuesta
					Galv\'anica de la piel y los eventos calendarizados en Microsoft Outlook para inferir y predecir periodos de estr\'es. Al detectar los periodos de
					carga de estr\'es, da recomendaciones de recalendarizaci\'on de actividades para balancear dichas cargas.
				\end{itemize}
		\chapter{Objetivos}
			\section{General}
				\begin{enumerate}
					\item Proponer un m\'etodo no intrusivo, basado en dispositivos vestibles de costo medio para detectar la ansiedad en cuidadores de pacientes con problemas mentales.
				\end{enumerate}
			\section{Espec\'ificos}
				\begin{enumerate}
					\item Encontrar que informaci\'on contextual es la m\'as adecuada para detectar periodos de ansiedad.
					\item Proponer una t\'ecnica basado en sensado oportun\'istico para la detecci\'on de ansiedad.
					\item Encontrar que dispositivos vestibles son los mas adecuados para recolectar la informaci\'on para detectar periodos de ansiedad.
					\item Desarrollar un prototipo de detecci\'on de periodos de ansiedad.
					\item Evaluar el prototipo.
				\end{enumerate}
			\section{Preguntas de investigaci\'on}
				\begin{enumerate}
					\item >C\'omo pueden los dispositivos vestibles de costo medio detectar la ansiedad en cuidadores de pacientes con problemas mentales?
				\end{enumerate}
		\chapter{Metodolog\'ia}
				\begin{enumerate}
					\item Revisi\'on de la literatura.
					\item Desarrollo de un escenario de aplicaci\'de estudio.
					\item Identificaci\'on de los tipos de se\~nales mas significativas para medir la ansiedad en el escenario seleccionado.
					\item Revisi\'on y selecci\'on de los dispositivos vestibles disponibles en el mercado para los tipos de datos seleccionados.
					\item Desarrollo de soluci\'on de software.
					\item Validaci\'on del software por medio de experimentaci\'on.
					\item Presentaci\'on de resultados.
				\end{enumerate}
		\chapter{Importancia de la investigaci\'on}
				Una de las finalidades de la tecnlog\'ia es ayudar en los problemas de los humanos. Los avances recientes en c\'omputo vestible muestran
				ser una poderosa herramienta para monitorear nuestros cuerpos, debido a que son f\'aciles de vestir, cuentan con diversos sensores y se pueden
				llevar con nosotros una buen parte del d\'ia. Ese monitoreo constante es \'util tambi\'en para mejorar la calidad de vida en personas con muchos periodos de ansiedad. %Escenarios antes descritos?
				
					\chapter{Limitaciones y suposiciones}
			Se tomar\'an en cuenta solamente la detecci\'on de ansiedad bajo ciertos escenarios, debido a que las situaciones en las que una persona puede sufrir ansiedad son muy variadas como para englobarlas en una sola soluci\'on tecnol\'ogica.
			La naturaleza de esta investigaci\'on requiere de dispositivos con sensores especializados como medidores de frecuencia cardi\'aca y resistencia galv\'anica de la piel o monitores de parpadeo montados en la cabeza entre otros. A pesar de que la tecnolog\'ia m\'ovil y vestible ha logrado una importante penetreaci\'on en el mercado, la l\'inea base de dispositivos no cuenta con dichos sensores. Al requerirse esta particularidad en los dispositivos, podr\'ian desprenderse subtareas como el contacto con empresas especializadas, espera de entrega, y entrenamiento en el desarrollo para estas tecnolog\'ias. Dichas tareas podr\'ian tomar un tiempo considerable en el proceso de investigaci\'on. Debido a esto, ser\'a importante organizar los tiempos muertos para evitar retrasos en la finalizaci\'on del estudio.
			Por otro lado, el uso de dispositivos que a\'un no ingresan al mercado t\'ipico mexicano podr\'ia no representar el uso cotidiano del que la visi\'on del  c\'omputo ubicuo habla. Sin embargo, buscar\'an estrategias para combinar y adaptar la tecnolog\'ia disponible para lograr una situaci\'on mas realista.
		\chapter{Contribuci\'on al conocimiento}
			El desarrollo de esta investgiaci\'on contribuir\'a a la interacci\'on humano computadora en forma de una propuesta de model que permita la inferencia de ansiedad utilizando dispositivos vestibles en un sensado oportunista. As\'i de como recomendaciones para el dise\~no de aplicaciones futuras que utilicen este modelos.
		\chapter{Calendario de actividades}
		%1 StressSense: detecting Stress in Unconstraner Acoustic Environments using smartphones
	%2 A thoerical approach to stress and self-efficacy
	%3 Stress and anxiety, 	Department of Psychophysiological Nursing, School of Nursing, University of Maryland, Baltimore.The Nursing Clinics of North America [1990, 25(4):935-943]
	%4 AutoSense: Unobtrusively Wearable Sensor Suite for Inferring the Onset, Causality, and Consequences of Stress in the Field
	%5
	%6 stress@work
	%7 FEEL: frequent EDA and Event Logging
	\newpage
	\addcontentsline{toc}{chapter}{\normalsize\expandafter{Referencias}}
	{\normalsize
		\bibliographystyle{cicese}
		\bibliography{proposal}
	}
	\end{document}
